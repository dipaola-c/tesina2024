% \comando{entorno}


%CLASE
\documentclass[a4paper, 10pt]{article}

%PREAMBULO
%Paquetes
\usepackage[spanish]{babel}
\usepackage[utf8]{inputenc}
\usepackage[T1]{fontenc}
%\usepackage{chicago}
\usepackage[backend=biber]{biblatex}
\addbibresource{ref.bib}




%DATOS
\title{Título}
\author{Autor}
\date{Fecha}

%CUERPO
\begin{document}

    \maketitle %repite DATOS

    %Abstract
    \begin{abstract}
    breve descrip
    \end{abstract}


    % Tabla de contenidos
    \tableofcontents %indice

    \section{Sección primera}
    Texto de la sección.
    \\ nueva línea dentro del mismo párrafo

    Respeta el enter? y da sangría pero no espaciado

    \begin{flushright}
    Este es el tercer párrafo del documento y aparece justificado a la derecha, es decir alineado con el margen derecho del documento. Obsérvese que no todas las líneas empiezan a la misma altura.
    \end{flushright}
    \begin{center}
    Este es el último párrafo del documento y aparece justificado en el centro entre los márgenes del documento. Obsérvese que ahora las líneas no empiezan ni terminan a la misma altura.
    \end{center}

    \textbf{negrita}
    \textit{itálica}
    \emph{enfatiza}
    \underline{subrayado}

    %El logotipo de latex es $\LaTeX$.\footnote{Fue creado por Leslie Lamport}

    El principal libro sobre latex es, aunque también
    es muy interesante el artículo \cite{borbonlatex2022}

    \subsection{Subsección primera}
    Texto de la subsección.
    14

    \subsubsection{Subsubsección primera}
    Bla


    % Encabezado de subsección sin numerar, excluido de la tabla de contenidos
    \subsection*{Subsección segunda}
    Texto de la subsección.


%\bibliographystyle{chicago}
%\bibliography{ref}
\nocite{*}
\printbibliography
%FIN CUERPO pdflatex main.tex
\end{document}
